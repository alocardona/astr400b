\documentclass[preprint2]{aastex7}
\newcommand{\vdag}{(v)^\dagger}
\newcommand\aastex{AAS\TeX}
\newcommand\latex{La\TeX}

\begin{document}

\title{M33's Stellar Stream Kinematic Evolution during Milkdromeda Merger}
\author[orcid=0000-0000-0000-0001,sname='Alondra']{Alondra Cardona}
\affiliation{The University of Arizona, Department of Astronomy and Steward Observatory}
\email{alondrac@arizona.edu}

\begin{abstract}

This is the abstract...for now...

\end{abstract}

\keywords{{Satellite Galaxy} --- {Local Group} --- {Jacobi Radius} --- {Tidal Tails} --- {Tidal Stripping} --- {Velocity Dispersion} --- {Galaxy Evolution}}


\section{Introduction} 
The tidal stripping of stellar mass that occurs when a galaxy is being tidally disrupted often creates distinct stellar stream structures. Tidal streams and stellar streams will be used interchangeably throughout this work. In the case of a satellite galaxy whose host galaxy is merging with another, the satellite's tidal debris will be pulled towards the more massive merging system. The kinematics of these stellar streams are also related to galaxy evolution, revealing how the evolutionary history of a satellite galaxy is impacted by dynamical interactions with the other galaxies in the system. The evolving dynamics of M33's stellar streams throughout the Milkdromeda --Milky Way and Andromeda-- merger will be further studied in this work.\\
This topic is very important in understanding galaxy evolution because it helps contextualize how tidal disruptions in the form of stellar streams are created and evolved by kinematic interactions between galaxies. Studying the kinematic profiles of M33's stellar streams provides a better understanding of how a merger can impact the evolutionary history of a satellite galaxy\citep{Johnston_1999}. A \textbf{galaxy} can be defined as "a gravitationally bound collection of stars whose properties cannot be explained by a combination of baryons and Newton’s laws of gravity.", which implies the necessary presence of a dark matter halo\citep{Willman_2012}. Correspondingly, \textbf{galaxy evolution} refers to the way in which galaxy objects grow and change over time.\\
The leading theory on galaxy formation is described by the cold dark matter (CDM) cosmological model, which suggests that galaxies form through hierarchical merger events and accretion. The CDM model suggests that galaxies lie within dark matter halos that form through the accretion and disruption of low-mass subhalos. Therefore, it is very important to understand the dynamical mechanisms that govern the formation and evolution of tidal debris in the form of stellar streams. These are the products of the subhalo disruption processes believed to be responsible for galaxy formation\citep{Shipp_2023}. Fig. 1 shows the streams of a low-mass satellite simulated with gravitational acceleration (right panel) and without gravitational acceleration (left panel), which shows how certain dynamical mechanisms can impact stellar stream density and kinematic profiles.\\
Some of the open questions that exist within this topic include how the dark matter halo structure of the merger system relates to the satellite's stellar stream morphology. The role these stellar streams play in accretion events is also uncertain\citep{10.1093/mnras/sty2474}. More research is also being done to better understand the physics behind predicting stellar stream patterns and kinematic behavior \citep{Reshetnikov_2001}. \\
\begin{figure}[h]
\centering
    \includegraphics[width=1\linewidth]{Choi2007fig.png}
    \label{fig:enter-label}
    \caption{Streams of a low-mass satellite simulated with gravitational acceleration (right panel) and without gravitational acceleration (left panel)\citep{10.1111/j.1365-2966.2007.12313.x}. These plots are color coded by particle density and demonstrate how changes in kinematic conditions lead to different stream structures.}
\end{figure}
\section{This Project}
In this work, we will study the dynamics of M33's tidal streams throughout the Milkdromeda merger. We will take a closer look at how the disk stars' velocities change before, during, and after the merger. Stars in the tidal streams will be selected for this analysis.\\
This work will try to provide more context towards understanding the patterns and kinematic behavior of stellar stream structures on a satellite of a merger system.\\
Having a better understanding of these structures' kinematics is very important in the context of the CDM cosmological model. Being able to characterize stellar streams allows us to consider their role in unconventional star formation and galaxy evolution theories. 
\section{Methodology}
This work will be based off the simulation data for the Milkdromeda collision presented in \citep{Marel_Besla_2012}, which depicts the most likely fates of the Milky Way, Andromeda, and M33 galaxy system over the next few gigayears. This is a collisionless N-body simulation that predicts the behaviors of any given large number of particles (N) with some attributed mass in the system. The initial parameters used in the simulation and the number of particles in each galaxy component (halo, disk, and bulge) are defined in Table 1 of \citep{Marel_Besla_2012}.\\
In order to study the fine structures of M33's stellar streams, the high-resolution simulation data must be used. All plots and analysis in this work will solely consider disk particles, not taking into account how M33's halo behaves and interacts with the system throughout the merger (see Fig. 2). 
\begin{figure}[h]
\centering
    \includegraphics[width=2.1\linewidth]{M33streamplots.pdf}
    \label{fig:enter-label}
    \caption{Plots of M33 disk particles color coded by velocity magnitude during significant stages of the Milky Way-Andromeda merger. A) M33 at the present day, B) M33 during the Milky Way and Andromeda's  first close encounter C) M33 after first Milky Way and Andromeda close encounter, D) M33 during Milky Way and Andromeda's second close encounter, E) M33 when the Milky Way and Andromeda merge, F) M33 after only merger remnant remains. These plots suggest dynamically active tidal streams and tidal stripping in M33's disk during the Milkdromeda merger.}
\end{figure}

In order to examine M33's evolving streams, particles that make up those streams must be selected for through computing the Jacobi radius of the galaxy. By definition, the particles that are at or beyond the Jacobi radius will no longer be gravitationally bound to M33 anymore \citep{Read2005TheTS}. The Jacobi radius of M33 will be computed as stated in Eq. 1.
\begin{equation} \label{eq:1}
    R_j = r  \bigg( \frac{M_{sat}}{2 M_{host}(<r)} \bigg)^{1/3}
\end{equation}
\newline
where R$_{j}$ is the Jacobi radius, r is the ditance from one galaxy's center of mass to the other's, M$_{sat}$ is the mass of the satellite galaxy, and M$_{host}$(<r) is the mass of the host galaxy within the distance between the two.\\
In addition to the velocity dispersion plots showing how M33's streams evolve over time with velocity magnitude color coding, a plot of average stream particle velocity over time will also be created. A Jacobi radius over time plot will also be created to help us visualize what main parameters are changing throughout the\\
\newline
\newline
\newline
\newline
\newline
\newline
\newline
\newline
\newline
\newline
\newline
\newline
\newline
\newline
\newline
\newline
\newline
\newline
\newline
\newline
\newline
\newline
\newline
\newline
\newline
\newline
\newline
\newline
\newline\\
simulation. Although this work will not discuss mass loss rates, the effects of mass loss due to the tidal stripping that M33 will experience will be reflected through the changing Jacobi radius and average stream particle velocity.\\
From the plots produced in this work, I expect to see the velocity of stellar stream particles increase with time as M33 is gravitationally pulled towards the MW-M31 merger. The velocity values are expected to change during close encounters with the MW-M31 merger as particle and momentum distributions are rearranged throughout the system.\\

\section{Results}
\begin{figure}[h!]
\centering
    \includegraphics[width=0.9\linewidth]{VvsT.png}
    \label{fig:enter-label}
    \caption{Plot of the average velocity of stellar stream particles over time.}
\end{figure}


\pagebreak\\

\bibliography{references}{}
\bibliographystyle{aasjournalv7}


\end{document}

